%!TEX TS-program = lualatex
%!TEX encoding = UTF-8 Unicode

\documentclass[t]{beamer}

%%%% HANDOUTS For online Uncomment the following four lines for handout
%\documentclass[t,handout]{beamer}  %Use this for handouts.
%\usepackage{handoutWithNotes}
%\pgfpagesuselayout{3 on 1 with notes}[letterpaper,border shrink=5mm]
%	\setbeamercolor{background canvas}{bg=black!5}


%%% Including only some slides for students.
%%% Uncomment the following line. For the slides,
%%% use the labels shown below the command.
%\includeonlylecture{student}

%% For students, use \lecture{student}{student}
%% For mine, use \lecture{instructor}{instructor}


%\usepackage{pgf,pgfpages}
%\pgfpagesuselayout{4 on 1}[letterpaper,border shrink=5mm]

% FONTS
\usepackage{fontspec}
\def\mainfont{Linux Biolinum O}
\setmainfont[Ligatures={Common,TeX}, Contextuals={NoAlternate}, BoldFont={* Bold}, ItalicFont={* Italic}, Numbers={Proportional}]{\mainfont}
\setsansfont[Scale=MatchLowercase]{Linux Biolinum O} 
\usepackage{microtype}

\usepackage{graphicx}
	\graphicspath{%
	{/Users/mtaylor/pictures/ctl_img}
	{/Users/mtaylor/pictures/teach/163/lecture/}
	{/Users/mtaylor/pictures/teach/348/lectures/}
	}

\usepackage{amsmath,amssymb}

%\usepackage{units}

\usepackage{booktabs}
\usepackage{multicol}
%	\setlength{\columnsep=1em}

%\usepackage{textcomp}
%\usepackage{setspace}
\usepackage{tikz}
	\tikzstyle{every picture}+=[remember picture,overlay]

\definecolor{eveblue}{RGB}{80,152,242}

\mode<presentation>
{
  \usetheme{Lecture}
  \setbeamercovered{invisible}
  %\setbeamertemplate{items}[square]
  \setbeamertemplate{enumerate items}{\color{black}\insertenumlabel)}
}

%\usepackage{calc}
%\usepackage{hyperref}

\newcommand\HiddenWord[1]{%
	\alt<handout>{\rule{\widthof{#1}}{\fboxrule}}{#1}%
}


\begin{document}

{\usebackgroundtemplate{\includegraphics[width=\paperwidth]{engage_intro}}
\begin{frame}[t]

\vfilll

\tiny \href{https://pixnio.com/people/crowd/silhouette-audience-crowd-concert-music-people-music-stage-performance}{\textcolor{white}{Author unknown, Pixnio, public domain.}}
\end{frame}
}

%

{\usebackgroundtemplate{\includegraphics[width=\paperwidth]{classroom_smartphones}}
	\begin{frame}[t]
		
		\tinyfill \href{https://hechingerreport.org/will-giving-greater-student-access-smartphones-improve-learning/}{\textcolor{white}{Paul Barnwell, The Hechinger Report, \ccbyncsa{4.0}.}}
	\end{frame}
}


%
\begin{frame}[t]{\LARGE \bfseries \hfill Six steps to natural selection \hfill}

\Large
\begin{enumerate}
\item Genetic variation exists among individuals within a population of organisms

\bigskip

\item Genetic variation is passed to the offspring

\bigskip

\item All populations overproduce young. Only some will survive to reproduce

\end{enumerate}
\end{frame}
%

\begin{frame}[t]{\LARGE \bfseries \hfill Six steps to natural selection \hfill}

\Large

\begin{enumerate}
\item[\color{black}4)] The traits an individual inherits determines likelihood of surviving to reproduce

\bigskip

\item[\color{black}5)] Those with better chance of survival have better chance of producing more offspring

\bigskip

\item[\color{black}6)] Traits that enhance reproduction will become more prevalent in population

\rule{\linewidth}{0.4pt}

\centering Traits that enhance ability to reproduce are called 
\textbf{Adaptations}

\end{enumerate}
\end{frame}

%

{\usebackgroundtemplate{\includegraphics[width=\paperwidth]{jet_bored}}
\begin{frame}[t]

\vfilll

\tiny \textcolor{pink}{Jet is currently 17.5 years old.}
\end{frame}
}

%

\begin{frame}[t]{Michael Alley advocates the \highlight{assertion-evidence approach.}}
\begin{multicols}{2}

\hangpara Each slide contains one assertion supported by visual evidence.

\vspace{9\baselineskip}

\hangpara {\small \textcolor{gray}{Michael Alley is a Teaching Professor of Engineering at Penn State University.}}

\columnbreak

\includegraphics[height=0.75\textheight]{alley_cover}

\end{multicols}

\tinyfill \url{https://www.assertion-evidence.com/}

\end{frame}

%

\begin{frame}[t]{This sentence headline makes an assertion on the first topic in no more than two lines.}

\begin{center}
\includegraphics[width=\textwidth]{assertion_image_box_wide2}
\end{center}


\vspace{-\baselineskip}
\hangpara If necessary, identify key assumption or background for audience. Keep to two lines (18–24 point type).

\end{frame}

\begin{frame}[t]{This sentence headline makes an assertion\newline on another topic in no more than two lines.}

\begin{multicols}{2}
\begin{center}
\includegraphics[width=\linewidth]{assertion_image_box_square}
\end{center}

\columnbreak

Feature or call-out—no more than two lines.

\vspace{8\baselineskip}

Feature or call-out—no more than two lines.

\end{multicols}

\begin{tikzpicture}
\draw[thick] (6,1.5) -- (1,4.1);


\draw[thick] (6,6.2) -- (4.5,4.7);

\end{tikzpicture}

\end{frame}

%

\begin{frame}{Assertion-evidence increased student learning by 10 percentage points.}

\begin{multicols}{2}
\includegraphics[width=\linewidth]{comparison_results_table}

\columnbreak


Students learn better when the critical information is in the headline instead of buried in the bullets.

\end{multicols}

\tinyfill Alley et al. 2006. Tech. Comm. 53: 225

\end{frame}

%

\begin{frame}[t]{Fragments quickly outpace the blast wave and\\become the primary hazard to personnel.}

\centering

\includegraphics[width=0.8\textwidth]{fragments_small}

\vfilll

\hfill \includegraphics[width=0.3in]{arl_logo}
\end{frame}

%

%
{\usebackgroundtemplate{\includegraphics[width=\paperwidth]{fragments_big}}
\begin{frame}[t]{\bfseries Fragments quickly outpace the blast wave and become the primary hazard to personnel.}

\vfilll

\includegraphics[width=0.3in]{arl_logo}

\end{frame}
}

%

\begin{frame}[t]{\LARGE \bfseries \hfill Six steps to natural selection \hfill}

\Large

\begin{enumerate}
\item[\color{black}4)] The traits an individual inherits determines likelihood of surviving to reproduce

\bigskip

\item[\color{black}5)] Those with better chance of survival have better chance of producing more offspring

\bigskip

\item[\color{black}6)] Traits that enhance reproduction will become more prevalent in population

\rule{\linewidth}{0.4pt}

\centering Traits that enhance ability to reproduce are called 
\textbf{Adaptations}

\end{enumerate}
\end{frame}

%

{\usebackgroundtemplate{\includegraphics[width=\paperwidth]{darwin_natural_selection_quote}}
\begin{frame}
\end{frame}
}

{\usebackgroundtemplate{\includegraphics[width=\paperwidth]{darwin_natural_selection_quote_highlight}}
\begin{frame}
\end{frame}
}

{\usebackgroundtemplate{\includegraphics[width=\paperwidth]{overproduce_offspring}}
\begin{frame}
\tinyfill \href{https://en.m.wikipedia.org/wiki/File:Anemone_Fish_Eggs.jpg}{\textcolor{white}{Anemone Fish eggs, Silke Barron, Wikimedia, \ccby{2.0}}}
\end{frame}
}

%

{\usebackgroundtemplate{\includegraphics[width=\paperwidth]{individuals_vary}}
\begin{frame}
\tinyfill \href{https://upload.wikimedia.org/wikipedia/commons/thumb/7/7d/Harmonia_axyridis01.jpg/974px-Harmonia_axyridis01.jpg}{\textcolor{white}{Asian Ladybugs collage © Entomart, Wikimedia, \ccbync{}}}
\end{frame}
}

%

{\usebackgroundtemplate{\includegraphics[width=\paperwidth]{not_all_will_survive}}
\begin{frame}
\tinyfill \href{https://en.wikipedia.org/wiki/European_bee-eater\#/media/File:Merops_apiaster_04.jpg}{\textcolor{white}{European Bee-Eater, Pierre Dalous, Wikimedia, \ccbysa{3.0}}}
\end{frame}
}

%

{\usebackgroundtemplate{\includegraphics[width=\paperwidth]{traits_determine_survival}}
\begin{frame}
\tinyfill \textcolor{white}{Fence Lizard photo by Dustin Siegel, with permission.}
\end{frame}
}

%

{\usebackgroundtemplate{\includegraphics[width=\paperwidth]{more_offspring_baby_spiders}}
\begin{frame}
\tinyfill \textcolor{white}{Baby spiders, C. Frank Sharmer, Creative Commons.}
\end{frame}
}

%

{\usebackgroundtemplate{\includegraphics[width=\paperwidth]{better_traits_dandelion}}
\begin{frame}
\tinyfill \textcolor{white}{Dandelion photo by David Carillet, www.wildretina.com, Creative Commons.}
\end{frame}
}

%

% Cycle of Natural Selection
\begin{frame}[c]
	\begin{center}
	
	\vspace{1.5cm}
	\begin{tikzpicture}
			[scale=0.6, transform shape,
			myNode/.style={circle, inner sep=2mm, minimum size=4cm, draw=black, fill=blue3},
			myLine/.style={line width=2pt, draw=black!70!white, ->},
			textNode/.style={text width=3.5cm, align=flush center, font=\Large}]
		\node (adapt) at (0,0) {\textcolor{orange6}{\textbf{\Huge Adaptation}}};
		\foreach \r in {60,120,...,360} {%
			\begin{scope}[shift={(\r:6)}]
				\node[myNode] (n\r) {};
			\end{scope};
		};
		\draw [myLine] (n60) to (n360);
		\draw [myLine] (n360) to (n300);
		\draw [myLine] (n300) to (n240);
		\draw [myLine] (n240) to (n180);
		\draw [myLine] (n180) to (n120);
		\draw [myLine] (n120) to (n60);
	
		\node [textNode] at (n60) {Individuals \highlight{vary} in populations};
		\node [textNode] at (n360) {Populations \highlight{overproduce} offspring};
		\node [textNode] at (n300) {Not all \highlight{survive}};
		\node [textNode] at (n240) {\highlight{Traits} determine survival};
		\node [textNode] at (n180) {More survival, \highlight{more offspring}};
		\node [textNode] at (n120) {Better traits \highlight{spread} through population};
	
	\end{tikzpicture}
	
	\end{center}
\end{frame}

%

%
{
\usebackgroundtemplate{\includegraphics[width=\paperwidth]{engaging} }
\begin{frame}
\hfill \tiny \textcolor{black!30!white}{\href{https://www.flickr.com/photos/31767752@N08/2967148200}{jul13d0wn3s, Flickr, \ccby{2.0}}}

\end{frame}
}

%

{
\usebackgroundtemplate{\includegraphics[width=\paperwidth]{connect_the_dots}}
\begin{frame}[t,plain]
\end{frame}
}

%

{
\usebackgroundtemplate{\includegraphics[width=\paperwidth]{connect_the_dots_solved}}
\begin{frame}[t,plain]
\end{frame}
}

%

{
\usebackgroundtemplate{\includegraphics[width=\paperwidth]{sunflowers}
}
\begin{frame}[b,plain]
	\textcolor{white}{\tiny Sunflowers by Trey Ratcliff, Flickr, Creative Commons.}
\end{frame}
}

%

{
\usebackgroundtemplate{\includegraphics[width=\paperwidth]{spider_bee}
}
\begin{frame}[b,plain]
\tiny Goldenrod crab spider photo by Alvesgaspar, Wikimedia Commons.\hfill
\textcolor{white}{\href{https://www.youtube.com/watch?v=O9B_9XxZKJ8}{Link to Video}}
\end{frame}
}

%

{
\usebackgroundtemplate{\includegraphics[width=\paperwidth]{marine_iguana}}
\begin{frame}[b,plain]
	\tiny \textcolor{white}{Marine iguana on San Crist\'{o}bal Island, Galapagos by Les Williams, Flickr, Creative Commons.}\hfill
	\textcolor{white}{\href{https://www.youtube.com/watch?v=4tBWakZAGqU}{Link to Video}}
\end{frame}
}

%

{
\usebackgroundtemplate{\includegraphics[width=\paperwidth]{bird_paradise.jpg}}
\begin{frame}[b,plain]
	\tiny\textcolor{white}{Greater Bird of Paradise \textcopyright Tim Laman, All Rights Reserved. Used with permission. \hfill\href{http://www.youtube.com/watch?v=KIYkpwyKEhY}{Link to Video} }
\end{frame}
}

%

\begin{frame}[t]{Consider what you know about marine snow.}

\hangpara How do you think the vertical distribution of nutrients from the surface downward affects the vertical distribution of biomass and density of individuals?

\end{frame}

{\usebackgroundtemplate{\includegraphics[width=\paperwidth]{biomass_density_depth_student}}
\begin{frame}[t]{Plot biomass and density as a function of depth.}

\end{frame}}

%

{\usebackgroundtemplate{\includegraphics[width=\paperwidth]{biomass_density_depth}}
\begin{frame}[t]{Both biomass and density decrease with depth.}

\vspace*{10\baselineskip}

\hangpara Yet, the deep sea benthos is very diverse. Why?
\end{frame}}

%

{\usebackgroundtemplate{\includegraphics[width=\paperwidth]{cliffhanger}}
\begin{frame}

\vfilll

\tiny \href{https://flickr.com/photos/c0t0s0d0/3169502387/}{c0t0s0d0, Flickr, \ccbysa{2.0}}
\end{frame}
}

%


\begin{frame}[t]{\highlight{Stability-time} hypothesis. {\large (Sanders 1968)}}

	\hangpara What is the importance of stability?

	\hangpara What were the arguments against this hypothesis?

\end{frame}

%

\begin{frame}[t]{\highlight{Spatial heterogeneity} means the habitat is not uniform.}

	{\centering
	\includegraphics[width=0.75\textwidth]{spatial_heterogeneity_abyssal_plain}\par}

	\vspace*{-\baselineskip}\hangpara North Atlantic abyssal plain (4800 m depth). Translucent sea cucumbers are feeding on green detritus from surface.

\vskip0pt plus 1filll

\hfill\tiny\url{http://www.oceanlab.abdn.ac.uk/esonet/porcupine.php}
\end{frame}


%{\usebackgroundtemplate{\includegraphics[width=\paperwidth]{deep_sea_benthic_diversity}}
%\begin{frame}[b]
%
%\hfill\tiny After Rex 1981, Annu. Rev. Ecol. Syst. 122: 331.
%\end{frame}}


\begin{frame}[t]{Pull out a sheet of paper. In groups\dots}

	\vspace*{-\baselineskip}

	\begin{multicols}{2}

	\includegraphics[width=0.49\textwidth]{deep_sea_benthic_intermediate_diversity2}

	\columnbreak
	
	\hangpara Choose three factors from previous hypotheses or your own ideas.
	
	\hangpara Consider biomass, age, stability, competition, predation, sediment size, disturbance, patch dynamics, spatial heterogeneity, etc.

	\hangpara Develop a unique hypothesis by explaining how each factor contributes to benthic diversity. 
	
	\hangpara Explain how each factor contributes to diversity. 

	\end{multicols}

\end{frame}

%


\begin{frame}[t]{What is the scientific consensus?}

\hangpara\highlight{Spatial heterogeneity}

\vspace*{2\baselineskip}

\hangpara\highlight{Patch dynamics}

\vspace*{2\baselineskip}

\hangpara\highlight{Sediment size}

\end{frame}

%


{\usebackgroundtemplate{\includegraphics[width=\paperwidth]{dont_skip}}
\begin{frame}

\tinyfill \href{https://en.m.wikipedia.org/wiki/File:Upload_free_image_notext.svg}{Husky, Wikimedia Commons, \ccbysa{4.0}.}

\end{frame}
}

%


%

{\usebackgroundtemplate{\includegraphics[width=\paperwidth]{salt_grains}}
\begin{frame}
\tiny \hfill \textcolor{white}{Salt crystals grown on the International Space Station. NASA, public domain.}

\vfilll

\end{frame}
}


%

{\usebackgroundtemplate{\includegraphics[width=\paperwidth]{q_a}}
\begin{frame}

\vfilll

\tiny \textcolor{white}{
\href{https://www.songsimian.com/}{Mic JohnsonLP}, \href{https://flickr.com/photos/186095195@N02/49347213806/}{Flickr, \ccbysa{2.0}}}
\end{frame}
}

%

\begin{frame}[t]{Michael Alley advocates the \highlight{assertion-evidence approach.}}
\begin{multicols}{2}

\hangpara Each slide contains one assertion supported by visual evidence.

\vspace{9\baselineskip}

\hangpara {\small \textcolor{gray}{Michael Alley is a Teaching Professor of Engineering at Penn State University.}}

\columnbreak

\includegraphics[height=0.75\textheight]{alley_cover}

\end{multicols}

\tinyfill \url{https://www.assertion-evidence.com/}

\end{frame}

%

\begin{frame}
\begin{multicols}{3}

\includegraphics[width=\linewidth]{book_tufte}\\
Edward Tufte \\
\href{https://www.edwardtufte.com/tufte/}{www.edwardtufte.com}

\columnbreak

\includegraphics[width=\linewidth]{book_reynolds}\\
Garr Reynolds\\
\href{https://www.garrreynolds.com}{www.garrreynolds.com}\\
\href{https://www.presentationzen.com}{www.presentationzen.com}


\columnbreak

\includegraphics[width=\linewidth]{book_duarte}\\
Nancy Duarte\\
\href{https://www.duarte.com}{www.duarte.com}

\end{multicols}
\end{frame}

%
\end{document}
