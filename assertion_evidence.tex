%!TEX TS-program = lualatex
%!TEX encoding = UTF-8 Unicode

\documentclass[t]{beamer}

%%%% HANDOUTS For online Uncomment the following four lines for handout
%\documentclass[t,handout]{beamer}  %Use this for handouts.
%\usepackage{handoutWithNotes}
%\pgfpagesuselayout{3 on 1 with notes}[letterpaper,border shrink=5mm]
%	\setbeamercolor{background canvas}{bg=black!5}


%%% Including only some slides for students.
%%% Uncomment the following line. For the slides,
%%% use the labels shown below the command.
%\includeonlylecture{student}

%% For students, use \lecture{student}{student}
%% For mine, use \lecture{instructor}{instructor}


%\usepackage{pgf,pgfpages}
%\pgfpagesuselayout{4 on 1}[letterpaper,border shrink=5mm]

% FONTS
\usepackage{fontspec}
\def\mainfont{Linux Biolinum O}
\setmainfont[Ligatures={Common,TeX}, Contextuals={NoAlternate}, BoldFont={* Bold}, ItalicFont={* Italic}, Numbers={Proportional}]{\mainfont}
\setsansfont[Scale=MatchLowercase]{Linux Biolinum O} 
\usepackage{microtype}

\usepackage{graphicx}
	\graphicspath{%
	{/Users/mtaylor/pictures/ctl_img}}

\usepackage{amsmath,amssymb}

%\usepackage{units}

\usepackage{booktabs}
\usepackage{multicol}
%	\setlength{\columnsep=1em}

%\usepackage{textcomp}
%\usepackage{setspace}
\usepackage{tikz}
	\tikzstyle{every picture}+=[remember picture,overlay]

\definecolor{eveblue}{RGB}{80,152,242}

\mode<presentation>
{
  \usetheme{Lecture}
  \setbeamercovered{invisible}
  %\setbeamertemplate{items}[square]
  \setbeamertemplate{enumerate items}{\color{black}\insertenumlabel)}
}

%\usepackage{calc}
%\usepackage{hyperref}

\newcommand\HiddenWord[1]{%
	\alt<handout>{\rule{\widthof{#1}}{\fboxrule}}{#1}%
}


\begin{document}

{\usebackgroundtemplate{\includegraphics[width=\paperwidth]{engage_intro}}
\begin{frame}[t]

\vfilll

\tiny \href{https://pixnio.com/people/crowd/silhouette-audience-crowd-concert-music-people-music-stage-performance}{\textcolor{white}{Author unknown, Pixnio, public domain.}}
\end{frame}
}
%
{\usebackgroundtemplate{\includegraphics[width=\paperwidth]{sunflowers_light}}
\begin{frame}[t]{\LARGE \textcolor{white}{\hfill Goals For This Fine Day\hfill}}

\Large
\begin{itemize}

\onslide<2->{
\item[\textcolor{white}{\textbullet}] \textcolor{white}{I’ll explain the Assertion-Evidence structure of presentation slides, based on ideas developed by Michael Alley, a Teaching Professor of Engineering at Penn State University.}}

\onslide<3->{
\item[\textcolor{white}{\textbullet}] \textcolor{white}{He claims that the Assertion-Evidence structure leads to better student learning and retention compared to the bulleted lists found on typical PowerPoint slides.}}

\onslide<4>{
\item[\textcolor{white}{\textbullet}] \textcolor{white}{I’ll discuss some of the pros and cons of his structure because, while I think Alley’s ideas have merit (otherwise, I wouldn’t be talking about them), I think they can be improved.}}
\end{itemize}

\onslide<1->{
\tinyfill \href{https://www.flickr.com/photos/stuckincustoms/3428453770}{\textcolor{white}{Trey Ratcliff, Flickr, \ccbyncsa{2.0}}}}

\end{frame}
}

%

{\usebackgroundtemplate{\includegraphics[width=\paperwidth]{jet_bored}}
\begin{frame}[t]

\vfilll

\tiny \textcolor{pink}{Jet is currently 17.5 years old.}
\end{frame}
}
%
{\usebackgroundtemplate{\includegraphics[width=\paperwidth]{free_lunch}}
\begin{frame}[t]
\tinyfill \href{https://www.flickr.com/photos/hendricksphotos/6336669904/in/photolist-aDX6co-aDX5TL-6EfxPT-hpd1xu-25Wz6d-hpkoce}{\textcolor{white}{Zaidyn eating pizza, Daniel M.\ Hendricks, Flickr, \ccbysa{2.0}}}
\end{frame}
}
%

{\usebackgroundtemplate{\includegraphics[width=\paperwidth]{sunflowers_light}}
\begin{frame}[t]{\LARGE \textcolor{white}{\hfill Goals For This Fine Day\hfill}}

\Large
\begin{itemize}

\item[\textcolor{white}{\textbullet}] \textcolor{white}{I’ll explain the Assertion-Evidence structure of presentation slides, based on ideas developed by Michael Alley, a Teaching Professor of Engineering at Penn State University.}


\item[\textcolor{white}{\textbullet}] \textcolor{white}{He claims that the Assertion-Evidence structure leads to better student learning and retention compared to the bulleted lists found on typical PowerPoint slides.}


\item[\textcolor{white}{\textbullet}] \textcolor{white}{I’ll discuss some of the pros and cons of his structure because, while I think Alley’s ideas have merit (otherwise, I wouldn’t be talking about them), I think they can be improved.}
\end{itemize}

\tinyfill \href{https://www.flickr.com/photos/stuckincustoms/3428453770}{\textcolor{white}{Trey Ratcliff, Flickr, \ccbyncsa{2.0}}}
\end{frame}
}

%
%{\usebackgroundtemplate{\includegraphics[width=\paperwidth]{template_example}}
%\begin{frame}[t]
%\end{frame}
%}

%

%{\usebackgroundtemplate{\includegraphics[width=\paperwidth]{ppt_templates}}
%\begin{frame}[t]
%\end{frame}
%}

%

{\usebackgroundtemplate{\includegraphics[width=\paperwidth]{edward_tufte}}
\begin{frame}[t]
\end{frame}
}

%

{\usebackgroundtemplate{\includegraphics[width=\paperwidth]{robert_gaskins}}
\begin{frame}[t]
\end{frame}
}

%

%{\usebackgroundtemplate{\includegraphics[width=\paperwidth]{ichthy_example}}
%\begin{frame}[t]
%\end{frame}
%}

%

\begin{frame}[t]{Michael Alley advocates the \highlight{assertion-evidence approach.}}
\begin{multicols}{2}

\hangpara Each slide contains one assertion supported by visual evidence.

\columnbreak

\includegraphics[height=0.75\textheight]{alley_cover}

\end{multicols}

\vfilll

\tiny \url{https://www.assertion-evidence.com/}

\end{frame}

\begin{frame}[t]{This sentence headline makes an assertion on the first topic in no more than two lines.}

\begin{center}
\includegraphics[width=\textwidth]{assertion_image_box_wide}
\end{center}

\onslide<2>{
\vspace{-\baselineskip}
\hangpara If necessary, identify key assumption or background for audience. Keep to two lines (18–24 point type).}

\end{frame}

\begin{frame}[t]{This sentence headline makes an assertion\newline on another topic in no more than two lines.}

\begin{multicols}{2}
\begin{center}
\includegraphics[width=\linewidth]{assertion_image_box_square}
\end{center}

\columnbreak

Feature or call-out—no more than two lines.

\vspace{8\baselineskip}

Feature or call-out—no more than two lines.

\end{multicols}

\begin{tikzpicture}
\draw[thick] (6,1.5) -- (1,4.1);


\draw[thick] (6,6.2) -- (4.5,4.7);

\end{tikzpicture}

\end{frame}

%

{\usebackgroundtemplate{\includegraphics[width=\paperwidth]{iron_ore_orig}}
\begin{frame}[t]
\end{frame}
}

%
\begin{frame}[t]{Iron ores make up 5.6\% of the earth's crust
and account for 95\% of the metals used.}

\begin{multicols}{2}
\begin{center}
\includegraphics[width=0.7\linewidth]{iron_ore}

\onslide<2>{
\vspace{1\baselineskip}\includegraphics[width=0.5\linewidth]{iron}}

\end{center}




\columnbreak

\includegraphics[width=\linewidth]{iron_ore_distribution}
\end{multicols}




\begin{tikzpicture}
\node [text=blue7, align=center] at (8.8,5.6) {\bfseries Iron Ore Distribution};

\node [text=blue7, align=right] at (1.7,5.85) {\bfseries Iron Ore};

\draw [color=blue7, ultra thick] (9.6,4.55) ellipse (0.5cm and 0.4cm);

\draw[ultra thick, color=blue7] (4.1,4.55) -- (9.1, 4.55);

\onslide<2>{%
\draw[ultra thick, color=blue7, <-] (3,3) -- (3, 3.4);

\node [text width=3cm, text=blue7, align=left] at (2,0.0) {\bfseries Is strong\\and durable.};

\node [text width=5cm, text=blue7, align=left] at (6.6,0.0) {\bfseries Can be shaped,\\ sharpened, and welded.};

\draw[ultra thick, color=blue7] (1.5, 0.4) -- (2,0.95);

\draw[ultra thick, color=blue7] (4.7, 0.4) -- (3.9,0.95);

}



\end{tikzpicture}

\end{frame}

%

\begin{frame}[t]{Students learning from the transformed slide scored higher on an identical test question.}

\vspace{-\baselineskip}
\hangpara Question: How abundant is iron in the earth's crust?

\noindent \includegraphics[width=\linewidth]{iron_ore_recall}

\vspace{-\baselineskip}
\begin{multicols}{2}
\centering

Led to 59\% correct. 

\columnbreak

Led to 77\% correct.
\end{multicols}

\hangpara $p < $ 0.001


\tinyfill Alley et al.\ 2006

\end{frame}

%
{\usebackgroundtemplate{\includegraphics[width=\paperwidth]{tectonics_orig}}
\begin{frame}[t]
\end{frame}
}

\begin{frame}[t]{Plates move because of convection caused by heat from decay of radioactive elements in the mantle.}

\vspace{-0.5\baselineskip}

\includegraphics[width=\textwidth]{pot_crust}

\begin{tikzpicture}
\node [text=blue7] at (8,0.2) {\bfseries Uranium and Thorium are large “unstable” atoms.};

\node [text=blue7, text width=6cm] at (8,-1.2) {\bfseries Break down to produce smaller atoms, heat and radioactivity.};

\draw [ultra thick, <-, color = blue7] (8, -0.7) -- (8, 0);

\end{tikzpicture}

\vfilll

\tiny Miller 2004

\end{frame}

%

\begin{frame}[t]{Students learning from the transformed slide scored higher on an identical test question.}

\vspace{-\baselineskip}
\hangpara Question: Heat source for movement of lithospheric plates?

\noindent \includegraphics[width=\linewidth]{tectonics_recall}

\vspace{-\baselineskip}
\begin{multicols}{2}
\centering

Led to 54\% correct. 

\columnbreak

Led to 86\% correct.
\end{multicols}

\hangpara $p < $ 0.001


\tinyfill Alley et al.\ 2006

\end{frame}

%

\begin{frame}[t]{Assertion-evidence format significantly increased average student performance.}
\centering
\includegraphics[height=0.75\textheight]{comparison_results_table}


\tinyfill Alley et al. 2006. Tech. Comm. 53: 225
\end{frame}

\begin{frame}[t]{Fragments quickly outpace the blast wave and\\become the primary hazard to personnel.}

\centering

\includegraphics[width=0.8\textwidth]{fragments_small}

\vfilll

\hfill \includegraphics[width=0.3in]{arl_logo}
\end{frame}

%

%
{\usebackgroundtemplate{\includegraphics[width=\paperwidth]{fragments_big}}
\begin{frame}[t]{\bfseries Fragments quickly outpace the blast wave and become the primary hazard to personnel.}

\vfilll

\includegraphics[width=0.3in]{arl_logo}

\end{frame}
}

%

{\usebackgroundtemplate{\includegraphics[width=\paperwidth]{big_impact}}
\begin{frame}

\vfilll

\hfill \tiny \textcolor{black!30!white}{\href{https://www.flickr.com/photos/31767752@N08/2967148200}{jul13d0wn3s, Flickr, \ccby{2.0}}}
\tiny 
\end{frame}
}

%

{\usebackgroundtemplate{\includegraphics[width=\paperwidth]{process_visual_new}}
\begin{frame}

\vfilll

\tiny \textcolor{white}{\href{https://www.flickr.com/photos/rifqidahlgren/7838236492/}{Alyzza by Rifqi Dahlgren, Flickr, \ccbync{2.0}.}}
\end{frame}
}

%

\begin{frame}[t]{\phantom{Has the rate of loss of Arctic sea ice remained constant between 1980–2011?}}
\includegraphics[width=\linewidth]{arctic_sea_ice}

\tinyfill Arctic Sea Ice Extent, Grant Foster using data from the National Snow and Ice Center, 2011. 
\end{frame}

%

\begin{frame}[t]{Has the rate of loss of Arctic sea ice remained constant between 1980–2011?}
\includegraphics[width=\linewidth]{arctic_sea_ice}

\tinyfill Arctic Sea Ice Extent, Grant Foster using data from the National Snow and Ice Center, 2011. 
\end{frame}

%


{\usebackgroundtemplate{\includegraphics[width=\paperwidth]{iron_ore_orig}}
\begin{frame}[t]
\end{frame}
}

%

\begin{frame}[t]{Iron ores make up 5.6\% of the earth's crust
and account for 95\% of the metals used.}

\begin{multicols}{2}
\begin{center}
\includegraphics[width=0.7\linewidth]{iron_ore}

\vspace{1\baselineskip}\includegraphics[width=0.5\linewidth]{iron}

\end{center}

\columnbreak

\includegraphics[width=\linewidth]{iron_ore_distribution}
\end{multicols}

\begin{tikzpicture}
\node [text=blue7, align=center] at (8.8,5.6) {\bfseries Iron Ore Distribution};

\node [text=blue7, align=right] at (1.7,5.85) {\bfseries Iron Ore};

\draw [color=blue7, ultra thick] (9.6,4.55) ellipse (0.5cm and 0.4cm);

\draw[ultra thick, color=blue7] (4.1,4.55) -- (9.1, 4.55);

\draw[ultra thick, color=blue7, <-] (3,3) -- (3, 3.4);

\node [text width=3cm, text=blue7, align=left] at (2,0.0) {\bfseries Is strong\\and durable.};

\node [text width=5cm, text=blue7, align=left] at (6.6,0.0) {\bfseries Can be shaped,\\ sharpened, and welded.};

\draw[ultra thick, color=blue7] (1.5, 0.4) -- (2,0.95);

\draw[ultra thick, color=blue7] (4.7, 0.4) -- (3.9,0.95);

\end{tikzpicture}

\end{frame}

%

{\usebackgroundtemplate{\includegraphics[width=\paperwidth]{iron_mine}}
\begin{frame}{\textcolor{black}{\bfseries Iron ores make up 5.6\% of the earth’s crust and account for 95\% of the metals used.}}


\tinyfill \href{https://www.flickr.com/photos/indiawaterportal/9004444257/}{\textcolor{white}{Dalli-Rajhara iron ore mine, India. indiawaterportal.org, Flickr, \ccbyncsa{2.0}}}
\end{frame}
}


%

{\usebackgroundtemplate{\includegraphics[width=\paperwidth]{iron_working}}
\begin{frame}{\textcolor{white}{\bfseries Iron is strong, durable, and can be shaped, sharpened, and welded.}}

\tinyfill \href{https://upload.wikimedia.org/wikipedia/commons/thumb/8/8c/3_tourist_helping_artist_blacksmith_in_finland.JPG/1024px-3_tourist_helping_artist_blacksmith_in_finland.JPG}{\textcolor{white}{Finnish blacksmiths, wasapl, Wikimedia, \ccbysa{3.0}.}}
\end{frame}
}

{\usebackgroundtemplate{\includegraphics[width=\paperwidth]{coot_and_chicks}}
\begin{frame}

\tinyfill \href{https://www.flickr.com/photos/mikebaird/530816116/}{\textcolor{white}{American Coot and Chicks. Mike Baird, Flickr \ccby{2.0}}}
\end{frame}
}

%
\begin{frame}[t]{\LARGE \bfseries \hfill Six steps to natural selection \hfill}

\Large
\begin{enumerate}
\item Genetic variation exists among individuals within a population of organisms

\bigskip

\item Genetic variation is passed to the offspring

\bigskip

\item All populations overproduce young. Only some will survive to reproduce

\end{enumerate}
\end{frame}
%

\begin{frame}[t]{\LARGE \bfseries \hfill Six steps to natural selection \hfill}

\Large

\begin{enumerate}
\item[\color{black}4)] The traits an individual inherits determines likelihood of surviving to reproduce

\bigskip

\item[\color{black}5)] Those with better chance of survival have better chance of producing more offspring

\bigskip

\item[\color{black}6)] Traits that enhance reproduction will become more prevalent in population

\rule{\linewidth}{0.4pt}

\centering Traits that enhance ability to reproduce are called 
\textbf{Adaptations}

\end{enumerate}
\end{frame}

%

{\usebackgroundtemplate{\includegraphics[width=\paperwidth]{darwin_natural_selection_quote}}
\begin{frame}
\end{frame}
}

{\usebackgroundtemplate{\includegraphics[width=\paperwidth]{darwin_natural_selection_quote_highlight}}
\begin{frame}
\end{frame}
}

{\usebackgroundtemplate{\includegraphics[width=\paperwidth]{overproduce_offspring}}
\begin{frame}
\tinyfill \href{https://en.m.wikipedia.org/wiki/File:Anemone_Fish_Eggs.jpg}{\textcolor{white}{Anemone Fish eggs, Silke Barron, Wikimedia, \ccby{2.0}}}
\end{frame}
}

%

{\usebackgroundtemplate{\includegraphics[width=\paperwidth]{individuals_vary}}
\begin{frame}
\tinyfill \href{https://upload.wikimedia.org/wikipedia/commons/thumb/7/7d/Harmonia_axyridis01.jpg/974px-Harmonia_axyridis01.jpg}{\textcolor{white}{Asian Ladybugs collage © Entomart, Wikimedia, \ccbync{}}}
\end{frame}
}

%

{\usebackgroundtemplate{\includegraphics[width=\paperwidth]{not_all_will_survive}}
\begin{frame}
\tinyfill \href{https://en.wikipedia.org/wiki/European_bee-eater\#/media/File:Merops_apiaster_04.jpg}{\textcolor{white}{European Bee-Eater, Pierre Dalous, Wikimedia, \ccbysa{3.0}}}
\end{frame}
}

%

{\usebackgroundtemplate{\includegraphics[width=\paperwidth]{traits_determine_survival}}
\begin{frame}
\tinyfill \textcolor{white}{Fence Lizard photo by Dustin Siegel, with permission.}
\end{frame}
}

%

{\usebackgroundtemplate{\includegraphics[width=\paperwidth]{more_offspring_baby_spiders}}
\begin{frame}
\tinyfill \textcolor{white}{Baby spiders, C. Frank Sharmer, Creative Commons.}
\end{frame}
}

%

{\usebackgroundtemplate{\includegraphics[width=\paperwidth]{better_traits_dandelion}}
\begin{frame}
\tinyfill \textcolor{white}{Dandelion photo by David Carillet, www.wildretina.com, Creative Commons.}
\end{frame}
}

%

% Cycle of Natural Selection
\begin{frame}[c]
	\begin{center}
	
	\vspace{1.5cm}
	\begin{tikzpicture}
			[scale=0.6, transform shape,
			myNode/.style={circle, inner sep=2mm, minimum size=4cm, draw=black, fill=blue3},
			myLine/.style={line width=2pt, draw=black!70!white, ->},
			textNode/.style={text width=3.5cm, align=flush center, font=\Large}]
		\node (adapt) at (0,0) {\textcolor{orange6}{\textbf{\Huge Adaptation}}};
		\foreach \r in {60,120,...,360} {%
			\begin{scope}[shift={(\r:6)}]
				\node[myNode] (n\r) {};
			\end{scope};
		};
		\draw [myLine] (n60) to (n360);
		\draw [myLine] (n360) to (n300);
		\draw [myLine] (n300) to (n240);
		\draw [myLine] (n240) to (n180);
		\draw [myLine] (n180) to (n120);
		\draw [myLine] (n120) to (n60);
	
		\node [textNode] at (n60) {Individuals \highlight{vary} in populations};
		\node [textNode] at (n360) {Populations \highlight{overproduce} offspring};
		\node [textNode] at (n300) {Not all \highlight{survive}};
		\node [textNode] at (n240) {\highlight{Traits} determine survival};
		\node [textNode] at (n180) {More survival, \highlight{more offspring}};
		\node [textNode] at (n120) {Better traits \highlight{spread} through population};
	
	\end{tikzpicture}
	
	\end{center}
\end{frame}

%

{\usebackgroundtemplate{\includegraphics[width=\paperwidth]{salt_grains}}
\begin{frame}
\tiny \hfill \textcolor{white}{Salt crystals grown on the International Space Station. NASA, public domain.}

\vfilll

\end{frame}
}

%


\begin{frame}
\begin{multicols}{3}

\includegraphics[width=\linewidth]{book_tufte}\\
Edward Tufte \\
\href{https://www.edwardtufte.com/tufte/}{www.edwardtufte.com}

\columnbreak

\includegraphics[width=\linewidth]{book_reynolds}\\
Garr Reynolds\\
\href{https://www.garrreynolds.com}{www.garrreynolds.com}\\
\href{https://www.presentationzen.com}{www.presentationzen.com}


\columnbreak

\includegraphics[width=\linewidth]{book_duarte}\\
Nancy Duarte\\
\href{https://www.duarte.com}{www.duarte.com}

\end{multicols}
\end{frame}

%

{\usebackgroundtemplate{\includegraphics[width=\paperwidth]{q_a}}
\begin{frame}
\end{frame}
}

%

\end{document}
